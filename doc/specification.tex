\documentclass[english,titlepage,a4paper]{report}

\usepackage[utf8]{inputenc}
\usepackage[T1]{fontenc}
\usepackage[english]{babel}
\usepackage{color}

\title{Software Defined Radio Testbed}
\author{Vasilis Tsiligiannis} 

\begin{document}
\maketitle
\tableofcontents


\chapter{Introduction}
\section{Purpose}

The purpose of this document is to identify and create a complete set of Software and Hardware Requirements Specifications for implementing a Software Defined Radio Testbed.
The analysis shall contain an architectural overview and high level design and requirements specification which will be the basis to bootstrap the development and integration process.
Such an analysis is necessary in order to provide implementers a clear view and understanding of the system under development as well as defining standard interfaces for users of the system.

\section{Definitions and Conventions}

\section{Intended Audience}

The intended audience of this document are hardware and software developers who will implement the SDR Testbed described.

\section{Scope}

The Software Defined Radio Testbed system described in this document plans to become an essential tool for the research and development of SDR-based satellite communication systems.
This testbed will allow early experimentation and testing of new algorithms and configurations.
The implementation of an SDR Testbed system will span from software and hardware development to integration of various SDR toolchains and hardware components.

\section{References}


\chapter{Description}
\section{System Perspective}

SDR Testbed shall be an independent and totally self-contained system.
It is not part of a larger system.
Thus, no strict requirements for external interfaces are set.
Nevertheless, additional ready-made components not defined in this specification may be used to complete it.

\section{System Functions}

\section{Operating Environment}

\section{Design and Implementation Constraints}

\section{Documentation}

\section{Assumption and Dependencies}


\chapter{External Interfaces}
\section{User Interfaces}

\section{Software Interfaces}

\section{Hardware Interfaces}

\section{Communication Interfaces}


\chapter{Requirements}
\section{Functional Requirements}

\section{Performance Requirements}

\section{Security Requirements}


\chapter*{Appendix I: Glossary}
\addcontentsline{toc}{chapter}{Appendix I: Glossary}


\end{document}
