%% SDR Testbed System Requirements Specification
%% Copyright (C) 2019 Libre Space Foundation
%%
%% This work is licensed under a
%% Creative Commons Attribution-ShareAlike 4.0 International License.
%%
%% You should have received a copy of the license along with this
%% work.  If not, see <http://creativecommons.org/licenses/by-sa/4.0/>.

\documentclass[english,titlepage,a4paper]{report}

\usepackage[utf8]{inputenc}
\usepackage[T1]{fontenc}
\usepackage[english]{babel}
\usepackage{color}

\title{Software Defined Radio Testbed}
\author{Vasilis Tsiligiannis} 

\begin{document}
\maketitle
\tableofcontents


\chapter{Introduction}
\section{Purpose}

The purpose of this document is to identify and create a complete set of Software and Hardware Requirements Specifications for implementing a Software Defined Radio Testbed.
The analysis shall contain an architectural overview and high level design and requirements specification which will be the basis to bootstrap the development and integration process.
Such an analysis is necessary in order to provide implementers a clear view and understanding of the system under development as well as defining standard interfaces for users of the system.

\section{Definitions and Conventions}

\section{Intended Audience}

The intended audience of this document are hardware and software developers who will implement the SDR Testbed described.

\section{Scope}

The Software Defined Radio Testbed system described in this document plans to become an essential tool for the research and development of SDR-based satellite communication systems.
This testbed will allow early experimentation and testing of new algorithms and configurations.
The implementation of an SDR Testbed system will span from software and hardware development to integration of various SDR toolchains and hardware components.

\section{References}


\chapter{Description}
\section{System Functions}

A list of high-level requirements has been collected from various stakeholders.
The collection has been facilitated by the use of a requirements survey.
Based on the survey results and the resources allocated for the project, it is decided that the following high-level functions shall be supported:
\begin{itemize}
\item importing and exporting of data for input and output of experiments
\item selection of various SDR device combinations
\item single TX and RX paths
\item selection of different attenuation levels between TX and RX device
\item telecommunication channel simulation
\item first-come-first-served job queuing
\end{itemize}

Additionally, multiple iterations of design and implementation are planned which may further enhance or extend the above functions.
Note, that the above list is not exhaustive.
A more detailed list is provided later in chapter \ref{chapter_4}.

\section{System Perspective}

As a whole, the SDR Testbed shall be an independent and totally self-contained system and not part of a larger system.
Thus, no strict requirements for external interfaces are set.
Nevertheless, additional ready-made components not defined in this specification may be used to complete it.

\section{Operating Environment}

The software operating environment will be based, to the maximum extent possible, on free and open source software.
All software components of SDR Testbed will be running on Linux OS.
It is suggested that a modern enterprise grade Linux distribution (e.g. CentOS) should be used for this purpose.
The services and application will be containerized to provide isolation and easier management.

The hardware operating environment can be based both on commercial and/or non-commercial hardware.
The testbed will be developed for x86-64 hardware platform, either server or COTS PC.
Nevertheless, provision has been made so that it will be possible to run on different architectures with minimal changes.
Most of the communication between hardware components will be over USB or IP network.

\section{Design and Implementation Constraints}

The following design and implementation constraints will be present on the SDR testbed:
\begin{itemize}
\item Jobs will be queued and executed sequentially.
  The testbed can only handle a single experiment at a time.
  Nevertheless, scaling-out will be possible i.e. by adding more SDR testbeds.
\item A limited number of SDR devices will be supported.
  Hardware restrictions place a limit on the number of devices which can be supported.
  A fixed number of USB and Ethernet ports will be available on the system for connecting SDR devices.
\item The duration of an experiment will be limited.
  Since the jobs will be executed sequentially, a limit on the duration of experiments will be placed.
\end{itemize}

\section{Documentation}

The documentation will consist of two parts:
\begin{itemize}
\item Users' documentation. This documentation will be the manual on how to use the testbed from users' point of view.
\item Developers' documentation. This documentation will be intended for developers on how to further extend or maintain the system.
\end{itemize}

\section{Assumption and Dependencies}

This system will be designed with the assumption that any experiment will not exceed the performance limits of the hardware.
This includes CPU power, RAM available, storage space, etc. as well as electrical characteristics (e.g. PSU or USB power available).
See chapter \ref{chapter_4} for more details.


\chapter{External Interfaces}
\section{User Interfaces}

The entry point of the user to the SDR testbed shall be a Git repository.
The user shall commit:
\begin{itemize}
\item the scenario for an experiment
\item raw data to be fed to the SDR devices
\item configuration of the SDR testbed hardware
\end{itemize}
Pushing the commits to a remote Git repository shall queue a job to be executed to the computer host where the SDR devices are attached.
After the execution and completion of the job, the host shall upload the results to the server managing the jobs.
The user shall be able to view and download the results of the experiment from the job server.

\section{Software Interfaces}

\section{Hardware Interfaces}

\section{Communication Interfaces}


\chapter{Requirements} \label{chapter_4}
\section{Functional Requirements}

\section{Performance Requirements}

\section{Security Requirements}


\chapter*{Appendix I: Glossary}
\addcontentsline{toc}{chapter}{Appendix I: Glossary}


\end{document}
